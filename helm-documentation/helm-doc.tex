% Created 2017-03-25 六 12:51
% Intended LaTeX compiler: pdflatex
\documentclass[11pt]{article}
\usepackage[utf8]{inputenc}
\usepackage[T1]{fontenc}
\usepackage{graphicx}
\usepackage{grffile}
\usepackage{longtable}
\usepackage{wrapfig}
\usepackage{rotating}
\usepackage[normalem]{ulem}
\usepackage{amsmath}
\usepackage{textcomp}
\usepackage{amssymb}
\usepackage{capt-of}
\usepackage{hyperref}
\author{n}
\date{\today}
\title{}
\hypersetup{
 pdfauthor={n},
 pdftitle={},
 pdfkeywords={},
 pdfsubject={},
 pdfcreator={Emacs 25.1.1 (Org mode 9.0.5)}, 
 pdflang={English}}
\begin{document}

\tableofcontents

\section{Helm Generic Help}
\label{sec:orgadf110a}

‘helm’ is an Emacs framework for incremental
completions and narrowing selections.

Helm narrows the list of candidates as the pattern is typed and
updates the list in a live feedback.

Helm accepts multiple patterns (entered with a space between patterns).
Helm support also fuzzy matching in some places when specified.

Helm uses familiar Emacs navigation keys to move up and down the list,
however some keybindings maybe confusing for new users, here are some:

‘RET’ selects the candidate from the list and execute default action
on it, exiting helm session.

‘C-j’ execute the default action
but without exiting helm session, it may be not available in some places.

‘C-z’ show you a list of actions
available on current candidate or all marked candidates, this maybe
surprising for new helm users that expect
‘C-z’ for completions and have not
realized they are already completing something as soon as helm is
started! See \href{https://github.com/emacs-helm/helm/wiki\#helm-completion-vs-emacs-completion}{Helm wiki}

\subsection{Helm mode}
\label{sec:org7891df8}

‘helm-mode’ allows you enabling helm completion in native emacs functions,
so when you turn on ‘helm-mode’ commands like e.g ‘switch-to-buffer’ will use
helm completion instead of the usual emacs completion buffer.

\subsubsection{What is helmized and not when ‘helm-mode’ is enabled ?}
\label{sec:orgf6b8e6d}

Helm is providing completion on all functions in emacs using ‘completing-read’
and derived and ‘completion-in-region’, it uses generic functions for this.

For the functions using ‘completing-read’ and derived e.g ‘read-file-name’ helm
have a user variable that allows controlling which function to use for a specific
emacs command, it is ‘helm-completing-read-handlers-alist’, it allows also
disabling helm completion for a specific command when the specified
function is nil.
See its documentation for more infos.

\subsubsection{Helm functions vs helmized emacs functions}
\label{sec:orga425187}

Sometimes you have helm functions that do the same completion as other
emacs vanilla helmized functions, e.g ‘switch-to-buffer’ and
‘helm-buffers-list’, you have to understand that the native helm
functions like ‘helm-buffers-list’ can receive new features, allow
marking candidates, have several actions and much more whereas the
emacs vanilla helmized functions have only a helm completion, one
action and no more what emacs provide for this function, it is the
intended behavior.

So generally you have better time using the native helm command generally
much more featured than the emacs function helmized than ‘helm-mode’.

\subsection{Helm Help}
\label{sec:orgfe9e884}

C-c ?		Shows this generic Helm help.

While in the help buffer, you have most of the regular keybindings
available in emacs buffers, the most important are shown in
minibuffer; However due to the implementation that do not use regular
emacs keymap (you are in a loop when reading help buffer) they are
hardcoded and not modifiable, here they are:

\begin{center}
\begin{tabular}{lll}
Key & Alternative keys & Command\\
\hline
C-v & Space next & Scroll up\\
M-v & b prior & Scroll down\\
C-s &  & Isearch forward\\
C-r &  & Isearch backward\\
C-a &  & Beginning of line\\
C-e &  & End of line\\
C-f & right & Forward char\\
C-b & left & Backward char\\
C-n & down & Next line\\
C-p & up & Previous line\\
M-a &  & Backward sentence\\
M-e &  & Forward sentence\\
M-f &  & Forward word\\
M-b &  & Backward word\\
M-> &  & End of buffer\\
M-< &  & Beginning of buffer\\
C-<SPACE> &  & Toggle mark\\
TAB &  & Org cycle\\
M-<TAB> &  & Toggle visibility\\
M-w &  & Copy region\\
q &  & Quit\\
\end{tabular}
\end{center}


\subsection{Helm’s Basic Operations and Default Key Bindings}
\label{sec:org3b736a8}

\begin{center}
\begin{tabular}{lll}
Key & Alternative Keys & Command\\
\hline
C-p & Up & Previous Line\\
C-n & Down & Next Line\\
M-v & PageUp & Previous Page\\
C-v & PageDown & Next Page\\
Enter &  & Execute first (default) action / Select\\
M-< &  & First Line\\
M-> &  & Last Line\\
C-M-S-v & M-PageUp, C-M-y & Previous Page (other-window)\\
C-M-v & M-PageDown & Next Page (other-window)\\
Tab & C-i & Show action list\\
Left &  & Previous Source\\
Right & C-o & Next Source\\
C-k &  & Delete pattern (with prefix arg delete from point to end)\\
C-j & C-z & Persistent Action (Execute and keep helm session)\\
\end{tabular}
\end{center}

\subsection{Shortcuts For nth Action}
\label{sec:orgd91bdeb}

f1-12: Execute nth Action where n is 1 to 12.

\subsection{Shortcuts for executing Default Action on the nth candidate}
\label{sec:orgf2f78b5}

C-x <n> => executes default action on number <n> candidate before currently selected candidate.

C-c <n> => executes default action on number <n> candidate after current selected candidate.

n is limited only to 1 through 9. For larger jumps use other
navigation keys. Also note that Helm candidates list by default
do not display line numbers. Line numbers can be enabled with the
linum-relative package.

\subsection{Marked candidates}
\label{sec:org9992a8d}

You can mark candidates to execute an action on them instead
of the current selected candidate only (See binding below).
Most Helm actions operate on marked candidates unless marking candidates
is prevented explicitely for a specific source.

\subsection{Follow candidates}
\label{sec:org0f20f83}

You can execute automatically an action specified in the source as persistent-action
while moving up and down in helm-window or while updating the list of candidates by
turning on ‘helm-follow-mode’ while in helm.
The follow behavior will be saved and used in next emacs sessions when ‘helm-follow-mode-persistent’
is non-nil.

\subsection{Frequently Used Commands}
\label{sec:orga625018}

C-t		Toggle vertical/horizontal split on first hit and swap helm window on second hit.
C-x C-f		Drop into ‘helm-find-files’.
C-c C-k		Kill display value of candidate and quit (with prefix arg, kill the real value).
C-c C-y		Yank current selection into pattern.
C-c TAB		Copy selected candidate at point in current-buffer.
C-c C-f		Toggle automatic execution of persistent action.
<C-down>	Run persistent action and then select next line.
<C-up>		Run persistent action and then select previous line.
C-c C-u		Recalculate and redisplay candidates.
C-!		Suspend/reenable updates to candidates list.

\subsection{Global Commands}
\label{sec:orgbdaff22}

M-m r l revives the last ‘helm’ session.
Very useful for resuming previous Helm. Binding a key to this
command will greatly improve ‘helm’ interactivity especially
after an accidental exit.

\subsection{Debugging helm}
\label{sec:orgb45d447}

helm have a special variable called ‘helm-debug’, setting it to non-nil
will allow helm logging in a special outline-mode buffer.
Helm is resetting the variable to nil at end of each session.

A convenient command is bound to M-x helm-enable-or-switch-to-debug
and allow turning debugging to this session only.
To avoid accumulating log while you are typing your pattern, you can use
C-! to turn off updating, then when you
are ready turn it on again to start updating.

Once you exit your helm session you can access the debug buffer with ‘helm-debug-open-last-log’.
It is possible to save logs to dated files when ‘helm-debug-root-directory’
is set to a valid directory.

NOTE: Be aware that helm log buffers grow really fast, so use ‘helm-debug’ only when needed.

\subsection{Writing your own helm sources}
\label{sec:orga6f61b2}

It is easy writing simple sources for your own usage.
Basically in a call to ‘helm’ function, the sources are added as a
single source which can be a symbol or a list of sources in the :sources slot.
Sources can be builded with different eieio classes depending
what you want to do, for simplifying this several ‘helm-build-*’ macros are provided.
We will not go further here, see \href{https://github.com/emacs-helm/helm/wiki/Developing}{Helm wiki} for more infos.
Below simple examples to start with.

\begin{verbatim}
;; Candidates are stored in a list.
(helm :sources (helm-build-sync-source "test"
                 ;; A function can be used as well
                 ;; to provide candidates.
                 :candidates ’("foo" "bar" "baz"))
      :buffer "*helm test*")

;; Candidates are stored in a buffer.
;; Generally faster but doesn’t allow a dynamic updating
;; of the candidates list i.e the list is fixed on start.
(helm :sources (helm-build-in-buffer-source "test"
                 :data ’("foo" "bar" "baz"))
      :buffer "*helm test*")
\end{verbatim}

For more complex sources, See \href{https://github.com/emacs-helm/helm/wiki/Developing}{Helm wiki}
and the many examples you will find in helm source code.

\subsection{Helm Map}
\label{sec:org42d366b}
key             binding
---             -------

C-@             helm-toggle-visible-mark
C-c             Prefix Command
C-g             helm-keyboard-quit
TAB .. C-j      helm-execute-persistent-action
C-k             helm-delete-minibuffer-contents
C-l             helm-recenter-top-bottom-other-window
RET             helm-maybe-exit-minibuffer
C-n             helm-next-line
C-o             helm-next-source
C-p             helm-previous-line
C-q             ace-jump-helm-line
C-t             helm-toggle-resplit-and-swap-windows
C-v             helm-next-page
C-w             ??
C-x             Prefix Command
C-z             helm-select-action
ESC             Prefix Command
C-SPC           helm-toggle-visible-mark
C-!             helm-toggle-suspend-update
C-\{             helm-enlarge-window
C-\}             helm-narrow-window
<C-M-down>      helm-scroll-other-window
<C-M-up>        helm-scroll-other-window-down
<C-down>        helm-follow-action-forward
<C-up>          helm-follow-action-backward
<M-next>        helm-scroll-other-window
<M-prior>       helm-scroll-other-window-down
<XF86Back>      previous-history-element
<XF86Forward>   next-history-element
<down>          helm-next-line
<escape>        keyboard-escape-quit
<f1>            ??
<f10>           ??
<f11>           ??
<f12>           ??
<f13>           ??
<f2>            ??
<f3>            ??
<f4>            ??
<f5>            ??
<f6>            ??
<f7>            ??
<f8>            ??
<f9>            ??
<help>          Prefix Command
<left>          helm-previous-source
<next>          helm-next-page
<prior>         helm-previous-page
<right>         helm-next-source
<tab>           helm-execute-persistent-action
<up>            helm-previous-line

<help> m        helm-help

C-c C-f         helm-follow-mode
C-c TAB         helm-copy-to-buffer
C-c C-k         helm-kill-selection-and-quit
C-c C-u         helm-refresh
C-c C-y         helm-yank-selection
C-c -           helm-swap-windows
C-c 1           ??
C-c 2           ??
C-c 3           ??
C-c 4           ??
C-c 5           ??
C-c 6           ??
C-c 7           ??
C-c 8           ??
C-c 9           ??
C-c >           helm-toggle-truncate-line
C-c ?           helm-help

C-x C-b         helm-resume-list-buffers-after-quit
C-x C-f         helm-quit-and-find-file
C-x 1           ??
C-x 2           ??
C-x 3           ??
C-x 4           ??
C-x 5           ??
C-x 6           ??
C-x 7           ??
C-x 8           ??
C-x 9           ??
C-x b           helm-resume-previous-session-after-quit

C-M-a           helm-show-all-in-this-source-only
C-M-e           helm-display-all-sources
C-M-l           helm-reposition-window-other-window
C-M-v           helm-scroll-other-window
C-M-y           helm-scroll-other-window-down
M-SPC           spacemacs/helm-navigation-transient-state/body
M-(             helm-prev-visible-mark
M-)             helm-next-visible-mark
M-<             helm-beginning-of-buffer
M->             helm-end-of-buffer
M-U             helm-unmark-all
M-a             helm-mark-all
M-m             helm-toggle-all-marks
M-n             next-history-element
M-o             helm-previous-source
M-p             previous-history-element
M-v             helm-previous-page
M-s-SPC         spacemacs/helm-navigation-transient-state/body
C-M-S-v         helm-scroll-other-window-down

C-c C-l         helm-minibuffer-history

M-r             previous-matching-history-element
M-s             next-matching-history-element
  (that binding is currently shadowed by another mode)



\section{Helm Buffer}
\label{sec:org88785c2}

\subsection{Tips}
\label{sec:orge733318}

\subsubsection{Completion}
\label{sec:org2ff5d05}

\begin{enumerate}
\item Major-mode
\label{sec:org63c0550}

You can enter a partial name of major-mode (e.g. lisp, sh) to narrow down buffers.
To specify the major-mode, prefix it with "*" e.g. "*lisp".
If you want to match all buffers but the ones with a specific major-mode (negation),
prefix the major-mode with "!" e.g. "*!lisp".
If you want to specify more than one major-mode, separate them with ",",
e.g. "*!lisp,!sh,!fun" will list all buffers but the ones in lisp-mode, sh-mode and
fundamental-mode.

Enter then a space and a pattern to narrow down to buffers matching this pattern.

\item Search inside buffers
\label{sec:orge420310}

If you enter a space and a pattern prefixed by "@" helm will search for text matching
this pattern INSIDE the buffer (i.e not in the name of buffer).
NOTE that if you enter your pattern prefixed with "@" but escaped, helm will search a buffer
matching "@pattern" but will not search inside.

\item Search by directory name
\label{sec:orgc2e4453}

If you prefix the beginning of pattern with "/" the match will occur on directory name
of buffer, it is interesting to narrow down to one directory for example, subsequent string
entered after a space will match on buffer-name only.
Note that negation is not supported for matching on buffer-file-name.
You can specify more than one directory starting from helm v1.6.8

\item Fuzzy matching
\label{sec:org8646540}

Note that if ‘helm-buffers-fuzzy-matching’ is non--nil you will have
fuzzy matching on buffer names (not on directory name matching and major-mode though).
A pattern starting with "\^{}" will disable fuzzy matching and will match by exact regexp.

\item Examples
\label{sec:org2807853}

if I enter in pattern prompt:

"*lisp \^{}helm @moc"

helm will narrow down the list by selecting only buffers that are in lisp mode, start by helm
and match "moc" in their contents.

if I enter in pattern prompt:

"*lisp \^{}helm moc"

Notice there is no "@" this time
helm will look for lisp mode buffers starting by "helm" and have "moc" in their name.

if I enter in pattern prompt:

"*!lisp !helm"

helm will narrow down to buffers that are not in "lisp" mode and that do not match "helm"

if I enter in pattern prompt:

\emph{helm} w3

helm will narrow down to buffers that are in any "helm" subdirectory and matching w3.
\end{enumerate}

\subsubsection{Creating buffers}
\label{sec:org464e5c3}

When creating a new buffer use C-u to choose a mode for your buffer in a list.
This list is customizable, see ‘helm-buffers-favorite-modes’.

\subsubsection{Killing buffers}
\label{sec:org13849b1}

You have a command to kill buffer(s) and quit emacs and a command to kill buffers one by one
(no marked) without quitting helm.

You can run this persistent kill buffer command either with the regular
‘helm-execute-persistent-action’ called with a prefix arg (C-u C-j) or with its specific command
‘helm-buffer-run-kill-persistent’ see binding below.

\subsubsection{Meaning of colors and prefixes for buffers}
\label{sec:org1358681}

Remote buffers are prefixed with ’@’.
Red        => Buffer have its file modified on disk by an external process.
Indianred2 => Buffer exists but its file have been deleted.
Orange     => Buffer is modified and its file not saved to disk.
Italic     => A non--file buffer.

\subsection{Commands}
\label{sec:orgab7020d}

M-g s		Grep Buffer(s) works as zgrep too (C-u grep all buffers but non--file buffers).
C-s		Multi Occur buffer or marked buffers. (C-u toggle force searching current-buffer).
C-c o		Switch other window.
C-c C-o		Switch other frame.
C-x C-d		Browse Project from buffer.
C-M-\%		Query replace regexp in marked buffers.
M-\%		Query replace in marked buffers.
C-c =		Ediff current buffer with candidate.  If two marked buffers ediff those buffers.
M-=		Ediff merge current buffer with candidate.  If two marked buffers ediff merge those buffers.
C-=		Toggle Diff buffer with saved file without quitting.
M-U		Revert buffer without quitting.
C-x C-s		Save buffer without quitting.
M-D		Delete marked buffers and quit.
C-c d		Delete buffer without quitting helm.
M-m		Toggle all marks.
M-a		Mark all.
C-]		Toggle details.
C-c a		Show hidden buffers.
C-M-SPC		Mark all buffers with same type (color) than current.

\section{Helm Find Files}
\label{sec:orgb41b518}

\subsection{Tips}
\label{sec:org8745a5a}

\subsubsection{Navigation summary}
\label{sec:org855b74a}

For a better experience you can enable auto completion by setting
‘helm-ff-auto-update-initial-value’ to non-nil in your init file.
It is not enabled by default to not confuse new users.

\begin{enumerate}
\item Use ‘C-j’ (persistent action) on a directory to go down one level
\label{sec:orgc63659a}

On a symlinked directory a prefix arg will allow expanding to its true name.

\item Use ‘C-l’ on a directory to go up one level
\label{sec:org8f288d9}

\item Use ‘C-r’ to walk back the resulting tree of all the ‘C-l’ you did
\label{sec:orgf885a5b}

Note: The tree is reinitialized each time you enter a new tree with ‘C-j’
or by entering some pattern in prompt.
\end{enumerate}

\subsubsection{Find file at point}
\label{sec:org478d107}

Helm is using ‘ffap’ partially or completely to find file at point
depending on value of ‘helm-ff-guess-ffap-filenames’.
You can use full ‘ffap’ by setting this to non-nil (annoying).
Default value is nil which make ‘ffap’ working partially.

\begin{enumerate}
\item Find file at number line
\label{sec:org72c53e6}

With something like this at point:

\textasciitilde{}/elisp/helm/helm.el:1234

Helm will find this file at line number 1234.

\item Find url at point
\label{sec:org9720330}

When an url is found at point, helm expand to that url only.
Pressing RET jump to that url using ‘browse-url-browser-function’.

\item Find mail at point
\label{sec:org03f8b42}

When a mail address is found at point helm expand to this email address
prefixed by "mailto:". Pressing RET open a message buffer with this mail
address.
\end{enumerate}

\subsubsection{Quick pattern expansion}
\label{sec:org1ba65d2}

\begin{enumerate}
\item Enter ‘\textasciitilde{}/’ at end of pattern to quickly reach home directory
\label{sec:org4d827e8}

\item Enter ‘/’ at end of pattern to quickly reach root of your file system
\label{sec:org3a942c7}

\item Enter ‘./’ at end of pattern to quickly reach ‘default-directory’ (initial start of session)
\label{sec:orgca3e199}

If you are already in ‘default-directory’ this will move cursor on top.

\item Enter ‘../’ at end of pattern will reach upper directory, moving cursor on top
\label{sec:org7098469}

NOTE: This is different from using ‘C-l’ in that ‘C-l’ doesn’t move cursor on top but stays on previous
subdir name.

\item Enter ‘..name/’ at end of pattern start a recursive search of directories matching name under
\label{sec:org12bf978}
your current directory, see below the "Recursive completion on subdirectories" section for more infos.

\item Enter any environment var (e.g. ‘\$HOME’) at end of pattern, it will be expanded
\label{sec:org4a40fe0}

\item You can yank any valid filename after pattern, it will be expanded
\label{sec:orgb491369}

\item Special case with url’s at point
\label{sec:org78a88c1}

This have no effect at end of an url, you have first to kill pattern (‘C-k’)
before entering one of these quick expansions patterns.
\end{enumerate}

\subsubsection{Helm find files is fuzzy matching (start on third char entered)}
\label{sec:orgdb7ae2d}

e.g. "fob" or "fbr" will complete "foobar"
but "fb" will wait for a third char for completing.

\subsubsection{Use ‘C-u C-j’ to watch an image or ‘C-<down>’}
\label{sec:org4999aa9}

\subsubsection{‘C-j’ on a filename will expand in helm-buffer to this filename}
\label{sec:orgeda465b}

Second hit on ‘C-j’ will display buffer filename.
Third hit on ‘C-j’ will kill buffer filename.
NOTE: ‘C-u C-j’ will display buffer directly.

\subsubsection{To browse images directories turn on ‘helm-follow-mode’ and navigate with arrow keys}
\label{sec:orgd4fc9aa}

You can also use ‘helm-follow-action-forward’ and ‘helm-follow-action-backward’
(‘C-<down’ and ‘C-<left>’).

\subsubsection{You can turn off/on (toggle) autoupdate completion at any moment with ‘C-DEL’}
\label{sec:org697512e}

It is useful when auto completion is enabled and when trying to create a new file
or directory you want to prevent helm trying to complete what you are writing.
NOTE: On a terminal C-<backspace> may not work, use in this case C-c <backspace>.

\subsubsection{You can create a new directory and a new file at the same time}
\label{sec:org9723cae}

Just write the path in prompt and press ‘<RET>’.
e.g. You can create "\textasciitilde{}/new/newnew/newnewnew/my\(_{\text{newfile.txt}}\)".

\subsubsection{To create a new directory, add a "/" at end of new name and press <RET>}
\label{sec:org605accc}

\subsubsection{To create a new file just write the filename not ending with "/"}
\label{sec:org2df9c22}

\subsubsection{Recursive search from helm find files}
\label{sec:org7340168}

\begin{enumerate}
\item You can use helm browse project (see binding below)
\label{sec:org72a5663}

\begin{itemize}
\item With no prefix arg
If your current directory is under version control
with one of git or hg and you have installed helm-ls-git and/or helm-ls-hg
\url{https://github.com/emacs-helm/helm-ls-git.git}
\url{https://github.com/emacs-helm/helm-ls-hg}
you will see all your files under version control, otherwise
you will be back to helm-find-files.
\item With one prefix arg
You will see all the files under this directory
and other subdirectories (recursion) and this list of files will be cached.
\item With two prefix args
same but the cache will be refreshed.
\end{itemize}

\item You can start a recursive search with Locate or Find (See commands below)
\label{sec:org7eadaab}

With Locate you can use a local db with a prefix arg. If the localdb doesn’t already
exists, you will be prompted for its creation, if it exists and you want to refresh it,
give two prefix args.

Note that when using locate the helm-buffer is empty until you type something,
but helm use by default the basename of pattern entered in your helm-find-files session,
hitting M-n should just kick in the locate search with this pattern.
If you want to automatically do this add the ‘helm-source-locate’
to ‘helm-sources-using-default-as-input’.

\item Recursive completion on subdirectories
\label{sec:orgac2bf55}

Starting from the current directory you are browsing, it is possible
to have completion of all directories under here.
So if you are at "\emph{home/you/foo}" and you want to go to "\emph{home/you/foo/bar/baz/somewhere/else"
just type "/home/you/foo}..else" and hit ‘C-j’ or enter the final "/", helm will show you all
possibles directories under "foo" matching "else".
(Note that entering two spaces before "else" instead of two dots works also).

NOTE: Completion on subdirectories use locate as backend, you can configure
the command with ‘helm-locate-recursive-dirs-command’.
Because this completion use an index, you may not have all the recent additions
of directories until you update your index (with ‘updatedb’ for locate).

If for some reason you cannot use an index the find command from findutils can be
used for this, it will be slower of course, you will have to pass the basedir as
first argument of find and the subdir as the value for ’-(i)regex’ or ’-(i)name’
with the two format specs that are mandatory in ‘helm-locate-recursive-dirs-command’,
e.g "find \%s -type d -name ’*\%s*’" or "find \%s -type d -regex .*\%s.*\$".
\end{enumerate}

\subsubsection{Insert filename at point or complete filename at point}
\label{sec:orgf20dfc3}

On insertion (no completion, i.e nothing at point):

\begin{itemize}
\item ‘C-c i’         => insert absolute file name.
\item ‘C-u C-c i’     => insert abbreviate file name.
\item ‘C-u C-u C-c i’ => insert relative file name.
\end{itemize}

On completion:

\begin{itemize}
\item target starts by \textasciitilde{}/           => insert abbreviate file name.
\item target starts by / or [a-z]:/ => insert full path.
\item otherwise                     => insert relative file name.
\end{itemize}

\subsubsection{Using wildcard to select multiple files}
\label{sec:org6bb5a72}

Use of wilcard is supported to give a set of files to an action:

e.g. You can copy all the files with ".el" extension by using "*.el"
and then run your copy action.

You can do the same but with "**.el" (note the two stars),
this will select recursively all ".el" files under current directory.

Note that when copying recursively files, you may have files with same name
dispatched in the different subdirectories, so when copying them in the same directory
they would be overwrited. To avoid this helm have a special action called "backup files"
that have the same behavior as the command line "cp --backup=numbered", it allows you
copying for example many *.jpg files with the same name from different
subdirectories in one directory.
Files with same name are renamed like this: "foo.txt.\textasciitilde{}1\textasciitilde{}".
NOTE: This command is available only when ‘dired-async-mode’ is used.

NOTE: When using an action that involve an external backend (e.g. grep), using "**"
is not advised (even if it works fine) because it will be slower to select all your files,
you have better time letting the backend doing it, it will be faster.
However, if you know you have not many files it is reasonable to use this,
also using not recursive wilcard (e.g. "*.el") is perfectly fine for this.

This feature ("**") is activated by default with the option ‘helm-file-globstar’.
It is different than the bash shopt globstar feature in that to list files with a named extension
recursively you just have to specify e.g "\textbf{*.el" whereas in bash you have to specify "**/}.el"
which is not convenient as "**.el".
The directory selection with "**/" like bash shopt globstar option is not supported yet.

\subsubsection{Query replace regexp on filenames}
\label{sec:orgb6d5d01}

You can rename your files by replacing only part of filenames matching
a regexp.

e.g Rename recursively all files with ".JPG" extension to ".jpg":
Use the helm-file-globstar feature described in previous section by
entering at end of helm-find-files pattern "**.JPG", then hit ‘M-\%‘,
at first prompt enter "JPG", at second "jpg" and hit ‘RET‘.

Shortcut for basename without extension, only extension or all are available:

\begin{itemize}
\item Basename without extension => "\%."
\item Only extension             => ".\%"
\item All                        => "\%"
\end{itemize}

So in the example above you could do instead:
At first prompt enter ".\%", at second "jpg" and hit ‘RET‘.
Note that when using this instead of using "JPG" at first prompt, all extensions
will be renamed to "jpg" even if the extension of one of the files is e.g "png".

If you want to rename a serie of files from number 001 to 00x use $\backslash$# inside the replacement
string when you will be prompted for it.

e.g To rename the files "foo.jpg" "bar.jpg" and "baz.jpg"
    to "foo-001.jpg" "foo-002.jpg" "foo-003.jpg"

Use as replace regexp "\%." and as replacement string "foo-$\backslash$#".
Where "\%." is same as regexp ".*$\backslash$.jpg".

Note: You can do this with the serial renames actions you will find in the action menu
      for more sophisticated renaming, but using query replace regexp on filenames
      is a fast way for most common serial replacements.

Note also that unlike the serial rename actions the renamed files stay in their initial directory
and are not renamed to current directory, IOW use this ($\backslash$#) to rename files inside current directory.

In the second prompt (replace regexp with) shortcut for ‘upcase’, ‘downcase’ and ‘capitalize’
are available, respectively ‘\%u’, ‘\%d’ and ‘\%c’.

\subsubsection{Copying renaming asynchronously}
\label{sec:org821c118}

If you use async library (if you have installed helm from MELPA you do) you can enable
async for copying/renaming etc\ldots{} your files by enabling ‘dired-async-mode’.

Note that even when async is enabled, running a copy/rename action with a prefix arg
will execute action synchronously, it will follow also the first file of the marked files
in its destination directory.

\subsubsection{Bookmark your ‘helm-find-files’ session}
\label{sec:org619e63c}

You can bookmark your ‘helm-find-files’ session with ‘C-x r m’.
You can retrieve later these bookmarks easily by using M-x helm-filtered-bookmarks
or from the current ‘helm-find-files’ session just hitting ‘C-x r b’.

\subsubsection{Grep files from ‘helm-find-files’}
\label{sec:orge87082d}

You can grep individual files from ‘helm-find-files’ by using
‘C-s’.  This same command can
grep also recursively files from current directory when called with a
prefix arg, you will be prompted in this case for the file extensions
to use (grep backend) or the types of files to use (ack-grep backend),
see the ‘helm-grep-default-command’ documentation to setup this.
For compressed files or archives, use zgrep with
‘M-g z’.

Otherwise you can use other recursive commands like
‘M-g a’ or ‘M-g g’ that are much more
faster than using ‘C-s’ with a
prefix arg.  See ‘helm-grep-ag-command’ and
‘helm-grep-git-grep-command’ to setup this.

You can also use the gid shell command
‘M-g i’ from id-utils by creating
an ID index file with the ‘mkid’ shell command coming with the
id-utils package.

All these grep commands are using symbol at point as default pattern.
Note that default is a different thing than input (nothing is added to
prompt until you hit ‘M-n’).

\subsubsection{Setting up aliases in eshell allows you to setup powerful customized commands}
\label{sec:org2f0273d}

Adding eshell aliases to your ‘eshell-aliases-file’ or using the
‘alias’ command from eshell allows you to create personalized commands
not available in ‘helm-find-files’ actions and use them from ‘M-!’.
Example:
You want a command to uncompress your "*.tar.gz" files from ‘helm-find-files’:

\begin{enumerate}
\item Create an alias named untargz (or whatever) in eshell with the
\end{enumerate}
command "alias untargz tar zxvf \$*"

\begin{enumerate}
\item Now from ‘helm-find-files’ select your "*.tar.gz" file (you can
\end{enumerate}
mark files if needed) and hit ‘M-!’.

Note:

When using marked files with this, the meaning of prefix arg is quite
subtil: Say you have foo, bar and baz marked, when you run the alias
command ‘example’ on these files with no prefix arg it will loop on
the file list and run sequentially ‘example’ on each file:

example foo
example bar
example baz

However with a prefix arg it will apply ‘example’ on each file:

example foo bar baz

Of course the alias command should support this.

\subsubsection{Using Tramp with ‘helm-find-files’ to read remote directories}
\label{sec:org81cf4b4}

‘helm-find-files’ is working fine with tramp with however some limitations.

\begin{itemize}
\item By default filenames are not highlighted when working on remote directories,
\end{itemize}
this is controled by ‘helm-ff-tramp-not-fancy’ variable, if you change this,
expect helm becoming very slow unless your connection is super fast.

\begin{itemize}
\item Grepping files is not very well supported when used incrementally, see above
\end{itemize}
grep section.

\begin{itemize}
\item Locate is not working on remote directories.
\end{itemize}

\begin{enumerate}
\item Some reminders about Tramp syntax
\label{sec:orge70a9b9}

Not exhaustive, please read Tramp documentation.

\begin{itemize}
\item Connect to host 192.168.0.4 as foo user:

/scp:192.168.0.4@foo:

\item Connect to host 192.168.0.4 as foo user with port 2222:

/scp:192.168.0.4@foo\#2222:

\item Connect to host 192.168.0.4 as root using multihops syntax:

/ssh:192.168.0.4@foo|sudo:192.168.0.4:
\end{itemize}

Note: you can also use ‘tramp-default-proxies-alist’ when connecting often to
some hosts.

Prefer generally scp method unless using multihops (works only with ssh method)
specially when copying large files.

Note also that you have to hit once ‘C-j’ on top of directory at first connection
to complete your pattern in minibuffer.

\item Completing host
\label{sec:org71667e9}

As soon as you enter the first ":" after method e.g \texttt{/scp:} you will
have some completion about previously used hosts or from your \texttt{\textasciitilde{}/.ssh/config}
file, hitting ‘C-j’ or ‘right’ on a candidate will insert this host in minibuffer
without addind the ending ":".
As soon the last ":" is entered Tramp will kick in and you should see the list
of candidates a few seconds later.

When your connection fails, be sure to delete your tramp connection before retrying
with M-x ‘helm-delete-tramp-connection’.

\item Editing local files as root
\label{sec:org50f4c32}

Use the sudo method:

/sudo:host: or just /sudo::
\end{enumerate}

\subsection{Commands}
\label{sec:org21d7703}

C-x C-f		Run Locate (C-u to specify locate db, M-n insert basename of candidate)
C-x C-d		Browse project (‘C-u’ recurse, ‘C-u C-u’ recurse and refresh db)
C-c /		Run Find shell command from this directory.
C-s		Run Grep (C-u Recursive).
M-g p		Run Pdfgrep on marked files.
M-g z		Run zgrep (C-u Recursive).
M-g a		Run AG grep on current directory.
M-g g		Run git-grep on current directory.
M-g i		Run gid (id-utils).
M-.		Run Etags (C-u use thing-at-point ‘C-u C-u’ reload cache)
M-R		Rename File (C-u Follow).
M-\%		Query replace on marked files.
M-C		Copy File (C-u Follow).
M-B		Byte Compile File (C-u Load).
M-L		Load File.
M-S		Symlink File.
M-H		Hardlink file.
M-D		Delete File.
M-K		Kill buffer candidate without quitting.
C-c d		Delete file without quitting.
M-e		Switch to Eshell.
M-!		Eshell command on file (C-u Apply on marked files, otherwise treat them sequentially).
C-c =		Ediff file.
M-=		Ediff merge file.
C-c i		Complete file name at point.
C-c o		Switch other window.
C-c C-o		Switch other frame.
C-c C-x		Open file with external program (C-u to choose).
C-c X		Open file externally with default tool.
M-l		Rotate Image Left.
M-r		Rotate Image Right.
C-l		Go down precedent directory.
M-p		Switch to last visited directories history.
C-c h		Switch to file name history.
M-i		Show file properties in a tooltip.
M-a		Mark all visibles candidates.
C-c DEL		Toggle auto expansion of directories.
M-U		Unmark all candidates, visibles and invisibles.
C-c C-a		Gnus attach files to message buffer.
C-c p		Print file, (C-u to refresh printers list).
C-\{		Enlarge helm window.
C-\}		Narrow helm window.
C-]		Toggle basename/fullpath.
C-c r		Find file as root.
C-x C-v		Find alternate file.
C-c @		Insert org link.

\section{Helm read file name}
\label{sec:org7d812f8}

\subsection{Tips}
\label{sec:org150a784}

If you are here, you are probably using a vanilla command like ‘find-file’
helmized by ‘helm-mode’, this is cool, but it is even better for your file
navigation to use ‘helm-find-files’ which is fully featured.

\subsubsection{Navigation}
\label{sec:org38bd0bf}

\begin{enumerate}
\item Enter ‘\textasciitilde{}/’ at end of pattern to quickly reach home directory
\label{sec:org699f436}

\item Enter ‘/’ at end of pattern to quickly reach root of your file system
\label{sec:orgb24c6c9}

\item Enter ‘./’ at end of pattern to quickly reach ‘default-directory’ (initial start of session)
\label{sec:org8e6c461}

If you are in ‘default-directory’ move cursor on top.

\item Enter ‘../’ at end of pattern will reach upper directory, moving cursor on top
\label{sec:org9c091fe}

NOTE: This different to using ‘C-l’ in that ‘C-l’ don’t move cursor on top but stay on previous
subdir name.

\item You can complete with partial basename (start on third char entered)
\label{sec:org34db79f}

E.g. "fob" or "fbr" will complete "foobar"
but "fb" will wait for a third char for completing.
\end{enumerate}

\subsubsection{Persistent actions}
\label{sec:orgba91fcf}

By default ‘helm-read-file-name’ use the persistent actions of ‘helm-find-files’

\begin{enumerate}
\item Use ‘C-u C-j’ to watch an image
\label{sec:orgc9c865b}

\item ‘C-j’ on a filename will expand in helm-buffer to this filename
\label{sec:orgc4129bd}

Second hit on ‘C-j’ will display buffer filename.
Third hit on ‘C-j’ will kill buffer filename.
NOTE: ‘C-u C-j’ will display buffer directly.

\item To browse images directories turn on ‘helm-follow-mode’ and navigate with arrow keys
\label{sec:org1eb0936}
\end{enumerate}

\subsubsection{Delete characters backward}
\label{sec:org936e64c}

When you want to delete backward characters, e.g. to create a new file or directory,
autoupdate may keep updating to an existent directory preventing you from doing so.
In this case, type C-<backspace> and then <backspace>.
This should not be needed when copying/renaming files because autoupdate is disabled
by default in that case.
NOTE: On a terminal C-<backspace> may not work, use in this case C-c <backspace>.

\subsubsection{Create new directory and files}
\label{sec:orga9b5cba}

\begin{enumerate}
\item Create a new directory and a new file at the same time
\label{sec:orgc47d15d}

You can create a new directory and a new file at the same time, 
just write the path in prompt and press <RET>.
E.g. You can create "\textasciitilde{}/new/newnew/newnewnew/my\(_{\text{newfile.txt}}\)".

\item To create a new directory, add a "/" at end of new name and press <RET>
\label{sec:org22d2879}

\item To create a new file just write the filename not ending with "/"
\label{sec:org631537a}

\uline{NOTE}: File and directory creation work only in some commands (e.g ‘find-file’)
and will not work in other commands where it is not intended to return a file or a directory
(e.g ‘list-directory’).
\end{enumerate}

\subsection{Commands}
\label{sec:org1ee3fc5}

C-l		Go down precedent directory.
C-c DEL		Toggle auto expansion of directories.
C-]		Toggle basename.
C-c h		File name history.
C/M-RET		Maybe return empty string (unless ‘must-match’).
C-o		Goto next source.
M-o		Goto previous source.

\section{Helm Generic files}
\label{sec:org8454a25}

\subsection{Tips}
\label{sec:org420a85e}

\subsubsection{Locate}
\label{sec:org79223dc}

You can add after writing search pattern any of the locate command line options.
e.g. -b, -e, -n <number>\ldots{}etc.
See Man locate for more infos.

Some other sources (at the moment recentf and file in current directory sources)
support the -b flag for compatibility with locate when they are used with it.

When you enable fuzzy matching on locate with
‘helm-locate-fuzzy-match’, the search will be performed on basename
only for efficiency (so don’t add "-b" at prompt), as soon as you
separate your patterns with spaces, fuzzy matching will be disabled
and search will be done on the full filename.  Note that in multimatch
fuzzy is completely disabled, which mean that each pattern should be a
compliant regexp matching pattern (i.e "helm" will match "helm"
but "hlm" will NOT match "helm").

\subsubsection{Browse project}
\label{sec:orge0ea400}

When your directory is not under version control,
don’t forget to refresh your cache when files have been added/removed in your directory.

\subsubsection{Find command}
\label{sec:orgf7e9fdd}

Recursively search files using "find" shell command.

Candidates are all filenames that match all given globbing patterns.
This respects the options ‘helm-case-fold-search’ and
‘helm-findutils-search-full-path’.

You can pass arbitrary options directly to find after a "*" separator.
For example, this would find all files matching "book" that are larger
than 1 megabyte:

book * -size +1M

\subsection{Commands}
\label{sec:org4f1fffd}

C-]		Toggle basename.
C-s		Run grep (C-u recurse).
M-g z		Run zgrep.
C-c g		Run gid (id-utils).
M-g p		Run Pdfgrep on marked files.
M-C		Copy file(s)
M-R		Rename file(s).
M-S		Symlink file(s).
M-H		Hardlink file(s).
M-D		Delete file(s).
M-B		Byte compile file(s) (C-u load) (elisp).
M-L		Load file(s) (elisp).
C-=		Ediff file.
C-c =		Ediff merge file.
C-c o		Switch other window.
M-i		Show file properties.
M-.		Run etags (C-u use tap, C-u C-u reload DB).
M-x helm-yank-text-at-point		Yank text at point.
C-c C-x		Open file with external program (C-u to choose).
C-c X		Open file externally with default tool.
C-c @		Insert org link.

\section{Helm Grep}
\label{sec:orgda3683f}

\subsection{Tips}
\label{sec:org392a076}

\subsubsection{You can start grep with a prefix arg to recurse in subdirectories}
\label{sec:orgbd6f093}
However now that helm support git-grep and AG, you have better time
using one of those for your recursives search.

\subsubsection{You can use wild card when selecting files (e.g. *.el)}
\label{sec:org44b0553}

\subsubsection{You can grep in many differents directories by marking files or wild cards}
\label{sec:org0376147}

\subsubsection{You can save your results in a ‘helm-grep-mode’ buffer, see commands below}
\label{sec:org05228b5}

Once in this buffer you can use emacs-wgrep (external package not bundled with helm)
to edit your changes.

\subsubsection{Helm grep is supporting multi matching starting from version 1.9.4.}
\label{sec:org50f9d72}
Just add a space between each pattern like in most helm commands.

\subsubsection{Important}
\label{sec:orgbc367ff}

Grepping on remote file will work only with grep, not ack-grep, but it is
anyway bad supported as tramp doesn’t support multiple process running in a
short delay (less than 5s actually) among other things,
so I strongly advice hitting ‘C-!’ (i.e suspend process)
before entering anything in pattern, and hit again ‘C-!’ when
your regexp is ready to send to remote process, even if helm is handling
this by delaying each process at 5s. 
Or even better don’t use tramp at all and mount your remote file system on SSHFS.

\section{Helm Gid}
\label{sec:org690a930}

\subsection{Tips}
\label{sec:org303ff94}

Helm gid read the database created with the ‘mkid’ command from id-utils.
The name of the database file can be customized with ‘helm-gid-db-file-name’, it
is usually "ID".
Helm Gid use the symbol at point as default-input.
You have access to this command also from ‘helm-find-files’ which allow you to
navigate to another directory to consult its database.

NOTE: Helm gid support multi matches but only the last pattern entered will be
highlighted due to the lack of \texttt{-{}-color} support in GID itself.

\section{Helm AG}
\label{sec:org7254f8c}

\subsection{Tips}
\label{sec:orgfd604cf}

Helm AG is different from grep or ack-grep in that it works on a directory and not
a list of files.
You can ignore files and directories by using a ".agignore" file, local to directory
or global when placed in home directory (See AG man page for more infos).
This file supports same entries as what you will find in ‘helm-grep-ignored-files’ and
‘helm-grep-ignored-directories’.
As always you can access helm AG from ‘helm-find-files’.

Starting at version 0.30 AG allow providing one or more TYPE argument on its command line.
Helm provide completion on these TYPES arguments when available with your AG version,
Use a prefix argument when starting helm ag session to get this completion.
NOTE: You can mark several types to match in your ag query, however on the first versions of
AG providing this, only one type was allowed, so in this case the last marked will take effect.

\section{Helm git-grep}
\label{sec:orgf316f0e}

Helm git-grep is searching from current directory
(i.e default-directory or the directory currently browsed by helm-find-files).
If this current directory is a subdirectory of project and you want to match
also upper directories (i.e the whole project) use a prefix arg.

\subsection{Commands}
\label{sec:org91e22c4}

<M-down>		Next File.
<M-up>		Precedent File.
M-x helm-yank-text-at-point		Yank Text at point in minibuffer.
C-c o		Jump other window.
C-c C-o		Jump other frame.
<left>		Run default action (Same as RET).
C-x C-s		Save to a ‘helm-grep-mode’ enabled buffer.

\section{Helm PdfGrep Map}
\label{sec:org0908473}

\subsection{Commands}
\label{sec:org7bd22b3}

<M-down>		Next File.
<M-up>		Precedent File.
M-x helm-yank-text-at-point		Yank Text at point in minibuffer.

\section{Helm Etags Map}
\label{sec:org924bd75}

\subsection{Commands}
\label{sec:org9151ab4}

<M-down>		Next File.
<M-up>		Precedent File.
M-x helm-yank-text-at-point		Yank Text at point in minibuffer.

\section{Helm Ucs}
\label{sec:org9b1bf59}

\subsection{Tips}
\label{sec:org5cff668}

Use commands below to insert unicode characters
in current-buffer without quitting helm.

\subsection{Commands}
\label{sec:org95120b4}

Uses keymap ‘helm-ucs-map’, which is not currently defined.

M-x helm-ucs-persistent-insert		Insert char.
M-x helm-ucs-persistent-forward		Forward char.
M-x helm-ucs-persistent-backward		Backward char.
M-x helm-ucs-persistent-delete		Delete char backward.

\section{Helm bookmark name}
\label{sec:org49ecbae}

\subsection{Commands}
\label{sec:org2555d57}

C-c o		Jump other window.
C-d		Delete bookmark.
M-e		Edit bookmark.
C-]		Toggle bookmark location visibility.

\section{Helm eshell on file}
\label{sec:org69a0aaf}

\subsection{Tips}
\label{sec:org9733f46}

\subsubsection{Passing extra args after filename}
\label{sec:orgc10c3b1}

Normally your command or alias will be called with file as argument. E.g.,

<command> ’candidate\(_{\text{file}}\)’

But you can also pass an argument or more after ’candidate\(_{\text{file}}\)’ like this:

<command> \%s [extra\(_{\text{args}}\)]

’candidate\(_{\text{file}}\)’ will be added at ’\%s’ and your command will look at this:

<command> ’candidate\(_{\text{file}}\)’ [extra\(_{\text{args}}\)]

\subsubsection{Specify many files as args (marked files)}
\label{sec:org4c1f52d}

E.g. <command> file1 file2 \ldots{}

Call ‘helm-find-files-eshell-command-on-file’ with one prefix-arg
Otherwise you can pass one prefix-arg from the command selection buffer.
NOTE: This is not working on remote files.

With two prefix-arg before starting or from the command selection buffer
the output is printed to your ‘current-buffer’.

Note that with no prefix-arg or a prefix-arg value of ’(16) (C-u C-u)
the command is called once for each file like this:

<command> file1 <command> file2 etc\ldots{}

\subsection{Commands}
\label{sec:org914973e}

Uses keymap ‘helm-esh-on-file-map’, which is not currently defined.


\section{Helm ido virtual buffers}
\label{sec:org6420dcb}

\subsection{Commands}
\label{sec:org91c0c1a}

C-c o		Switch other window.
C-c C-o		Switch other frame.
M-g s		Grep file.
M-g z		Zgrep file.
M-D		Delete file.
C-c C-x		Open file externally.

\section{Helm Moccur}
\label{sec:org0950258}

\subsection{Tips}
\label{sec:org87b887d}

\subsubsection{Matching}
\label{sec:orgb281e97}

Multiple regexp matching is allowed, just enter a space to separate your regexps.

Matching empty lines is supported with the regexp "\^{}\$", you will get the results
with only the buffer-name and the line number, you can of course save and edit these
results (i.e add text to the empty line) .

\subsubsection{Automatically matching symbol at point}
\label{sec:orga31fb17}

You can match automatically the symbol at point, but keeping
the minibuffer empty ready to write into.
This is disabled by default, to enable this you have to add ‘helm-source-occur’
and ‘helm-source-moccur’ to ‘helm-sources-using-default-as-input’.

\subsubsection{Jump to the corresponding line in the searched buffer}
\label{sec:org32a099e}

You can do this with ‘C-j’ (persistent-action), to do it repetitively
you can use ‘C-<up>’ and ‘C-<down>’ or enable ‘helm-follow-mode’ with ‘C-c C-f’.

\subsubsection{Saving results}
\label{sec:orgb21b816}

Same as with helm-grep, you can save the results with ‘C-x C-s’.
Of course if you don’t save your results, you can get back your session
with ‘helm-resume’.

\subsubsection{Refreshing the resumed session.}
\label{sec:org82aa6b9}

When the buffer(s) where you ran helm-(m)occur have been modified, you will be
warned of this with the buffer flashing to red, you can refresh the buffer by running
‘C-c C-u’.
This can be done automatically by customizing ‘helm-moccur-auto-update-on-resume’.

\subsubsection{Refreshing a saved buffer}
\label{sec:org9991f8c}

Type ‘g’ to update your buffer.

\subsubsection{Edit a saved buffer}
\label{sec:org58425a8}

First, install wgrep \url{https://github.com/mhayashi1120/Emacs-wgrep}
and then:

\begin{enumerate}
\item C-c C-p to edit the buffer(s).
\item C-x C-s to save your changes.
\end{enumerate}

Tip: Use the excellent iedit \url{https://github.com/tsdh/iedit}
to modify occurences in your buffer.

\subsection{Commands}
\label{sec:orgf52383e}

<M-down>		Next Buffer.
<M-up>		Precedent Buffer.
M-x helm-yank-text-at-point		Yank Text at point in minibuffer.
C-c o		Goto line in other window.
C-c C-o		Goto line in new frame.

\section{Helm Top}
\label{sec:org5b38240}

\subsection{Tips}
\label{sec:orgeb8a16b}

\subsection{Commands}
\label{sec:orgda7f26a}

Uses keymap ‘helm-top-map’, which is not currently defined.

M-x helm-top-run-sort-by-com		Sort by commands.
M-x helm-top-run-sort-by-cpu		Sort by cpu usage.
M-x helm-top-run-sort-by-user		Sort alphabetically by user.
M-x helm-top-run-sort-by-mem		Sort by memory.

\section{Helm Apt}
\label{sec:orge43f987}

\subsection{Tips}
\label{sec:org0833685}

\subsection{Commands}
\label{sec:org4b99bb6}

Uses keymap ‘helm-apt-map’, which is not currently defined.

M-x helm-apt-show-all		Show all packages.
M-x helm-apt-show-only-installed		Show installed packages only.
M-x helm-apt-show-only-not-installed		Show not installed packages only.
M-x helm-apt-show-only-deinstalled		Show deinstalled (not purged yet) packages only.>

\section{Helm elisp package}
\label{sec:orge642d4d}

\subsection{Tips}
\label{sec:org3091621}

\subsubsection{Compile all your packages asynchronously}
\label{sec:org5fa79ce}

When using async (if you have installed from MELPA you do), only helm, helm-core,
and magit are compiled asynchronously, if you want all your packages compiled async,
add to your init file:

(setq async-bytecomp-allowed-packages ’(all))

\subsubsection{Upgrade elisp packages}
\label{sec:org27c80c5}

On initial start (when emacs is fetching packages on remote), if helm find
package to upgrade it will start in the upgradables packages view showing the packages
availables to upgrade.
On further starts, you will have to refresh the list with ‘C-c C-u’, if helm find upgrades
you will have a message telling you some packages are available for upgrade, you can switch to
upgrade view (see below) to see what packages are available for upgrade or just hit ‘C-c U’.
to upgrade all.

To see upgradables packages hit <M-U>.

Then you can install all upgradables packages with the upgrade all action (‘C-c C-u’),
or upgrade only the specific packages by marking them (the new ones) and running
the upgrade action (visible only when there is upgradables packages).
Of course you can upgrade a single package by just running the upgrade action
without marking it (‘C-c u’ or RET) .

\textbf{WARNING} You are strongly advised to RESTART emacs after UPGRADING packages.

\subsubsection{Meaning of flags prefixing packages (Emacs-25)}
\label{sec:orgc9e3081}

\begin{itemize}
\item The flag "S" that prefix package names mean that this package is one of ‘package-selected-packages’.
\end{itemize}
This feature is only available with emacs-25.

\begin{itemize}
\item The flag "U" that prefix package names mean that this package is no more needed.
\end{itemize}
This feature is only available with emacs-25.

\subsection{Commands}
\label{sec:orgd45d18d}

Uses keymap ‘helm-el-package-map’, which is not currently defined.

M-x helm-el-package-show-all		Show all packages.
M-x helm-el-package-show-installed		Show installed packages only.
M-x helm-el-package-show-uninstalled		Show not installed packages only.
M-x helm-el-package-show-upgrade		Show upgradable packages only.
M-x helm-el-package-show-built-in		Show built-in packages only.
M-x helm-el-run-package-install		Install package(s).
M-x helm-el-run-package-reinstall		Reinstall package(s).
M-x helm-el-run-package-uninstall		Uninstall package(s).
M-x helm-el-run-package-upgrade		Upgrade package(s).
M-x helm-el-run-package-upgrade-all		Upgrade all packages upgradables.
M-x helm-el-run-visit-homepage		Visit package homepage.

\section{Helm M-x}
\label{sec:org56390eb}

\subsection{Tips}
\label{sec:org664c479}

\subsubsection{You can get help on any command with persistent action (C-j)}
\label{sec:orga06743f}

\subsubsection{Prefix Args}
\label{sec:org34e51cd}

When you want pass prefix args, you should pass prefix args AFTER starting ‘helm-M-x’,
you will see a prefix arg counter appearing in mode-line notifying you
the number of prefix args entered.

If you pass prefix args before running ‘helm-M-x’, it will be displayed in prompt,
then the first C-u after ‘helm-M-x’ will be used to clear that prefix args.

\section{Helm imenu}
\label{sec:org21a0b10}

\subsection{Tips}
\label{sec:orgd0f749c}

\subsection{Commands}
\label{sec:orgaeb8994}

Uses keymap ‘helm-imenu-map’, which is not currently defined.

M-x helm-imenu-next-section		Go to next section.
M-x helm-imenu-previous-section		Go to previous section.

\section{Helm colors}
\label{sec:orgb603168}

\subsection{Commands}
\label{sec:org907dc33}

Uses keymap ‘helm-color-map’, which is not currently defined.

M-x helm-color-run-insert-name		Insert the entry’name.
M-x helm-color-run-kill-name		Kill the entry’s name.
M-x helm-color-run-insert-rgb		Insert entry in RGB format.
M-x helm-color-run-kill-rgb		Kill entry in RGB format.

\section{Helm semantic}
\label{sec:org32acb58}

\subsection{Tips}
\label{sec:org63322a4}

\subsection{Commands}
\label{sec:orgd9a1e0b}

Uses keymap ‘helm-semantic-map’, which is not currently defined.


\section{Helm kmacro}
\label{sec:org6bc5c35}

\subsection{Tips}
\label{sec:org6fae82c}

\begin{itemize}
\item Start recording some keys with ‘f3’
\item Record new kmacro with ‘f4’
\item Start ‘helm-execute-kmacro’ to list all your macros.
\end{itemize}

Use persistent action to run your kmacro as many time as needed,
you can change of kmacro with ‘helm-next-line’ ‘helm-previous-line’.

NOTE: You can’t record keys running helm commands except ‘helm-M-x’ unless
you don’t choose from there a command using helm completion.

\subsection{Commands}
\label{sec:org6f8b9b3}

Uses keymap ‘helm-kmacro-map’, which is not currently defined.
\end{document}
